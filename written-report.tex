% Options for packages loaded elsewhere
\PassOptionsToPackage{unicode}{hyperref}
\PassOptionsToPackage{hyphens}{url}
%
\documentclass[
]{article}
\usepackage{lmodern}
\usepackage{amssymb,amsmath}
\usepackage{ifxetex,ifluatex}
\ifnum 0\ifxetex 1\fi\ifluatex 1\fi=0 % if pdftex
  \usepackage[T1]{fontenc}
  \usepackage[utf8]{inputenc}
  \usepackage{textcomp} % provide euro and other symbols
\else % if luatex or xetex
  \usepackage{unicode-math}
  \defaultfontfeatures{Scale=MatchLowercase}
  \defaultfontfeatures[\rmfamily]{Ligatures=TeX,Scale=1}
\fi
% Use upquote if available, for straight quotes in verbatim environments
\IfFileExists{upquote.sty}{\usepackage{upquote}}{}
\IfFileExists{microtype.sty}{% use microtype if available
  \usepackage[]{microtype}
  \UseMicrotypeSet[protrusion]{basicmath} % disable protrusion for tt fonts
}{}
\makeatletter
\@ifundefined{KOMAClassName}{% if non-KOMA class
  \IfFileExists{parskip.sty}{%
    \usepackage{parskip}
  }{% else
    \setlength{\parindent}{0pt}
    \setlength{\parskip}{6pt plus 2pt minus 1pt}}
}{% if KOMA class
  \KOMAoptions{parskip=half}}
\makeatother
\usepackage{xcolor}
\IfFileExists{xurl.sty}{\usepackage{xurl}}{} % add URL line breaks if available
\IfFileExists{bookmark.sty}{\usepackage{bookmark}}{\usepackage{hyperref}}
\hypersetup{
  pdftitle={Racial Disparities in Traffic Stops/Citations},
  pdfauthor={Keohane sQUAD: Chris Liang, Andrew Qin, Bob Qian, and Katie Nash},
  hidelinks,
  pdfcreator={LaTeX via pandoc}}
\urlstyle{same} % disable monospaced font for URLs
\usepackage[margin=1in]{geometry}
\usepackage{graphicx,grffile}
\makeatletter
\def\maxwidth{\ifdim\Gin@nat@width>\linewidth\linewidth\else\Gin@nat@width\fi}
\def\maxheight{\ifdim\Gin@nat@height>\textheight\textheight\else\Gin@nat@height\fi}
\makeatother
% Scale images if necessary, so that they will not overflow the page
% margins by default, and it is still possible to overwrite the defaults
% using explicit options in \includegraphics[width, height, ...]{}
\setkeys{Gin}{width=\maxwidth,height=\maxheight,keepaspectratio}
% Set default figure placement to htbp
\makeatletter
\def\fps@figure{htbp}
\makeatother
\setlength{\emergencystretch}{3em} % prevent overfull lines
\providecommand{\tightlist}{%
  \setlength{\itemsep}{0pt}\setlength{\parskip}{0pt}}
\setcounter{secnumdepth}{-\maxdimen} % remove section numbering

\title{Racial Disparities in Traffic Stops/Citations}
\author{Keohane sQUAD: Chris Liang, Andrew Qin, Bob Qian, and Katie Nash}
\date{2020-10-27}

\begin{document}
\maketitle

\hypertarget{introduction-and-data}{%
\subsection{Introduction and Data}\label{introduction-and-data}}

Our data is a census of individual police stops in Durham created by the
Stanford Open Policing project. The Stanford Open Policing Project
``{[}collects{]} and {[}standardizes{]} data on vehicle and pedestrian
stops from law enforcement departments across the
country''(\url{https://openpolicing.stanford.edu/}). We would like to
see if that same kind of racial bias is evident in police stops in
Durham. In doing so, we also wish to examine if other demographic
characteristics (such as sex or age) influence traffic stops. Our
general research question is the following: what is the relationship
between a subject's demographic attributes (sex, race, or age) and the
likelihood of being stopped by police in traffic in Durham?

We hypothesize that race and the likelihood of being stopped by police
in traffic in Durham are related, with black people representing
disproportionately more of the people being stopped relative to their
proportion within the population. We hypothesize that younger people
(roughly 18-30) have a disproportionately higher chance of being stopped
in traffic (not necessarily due to bias but other lurking variables,
such as inexperienced driving). We also hypothesize that sex has no
significant relationship with being stopped in traffic. To find the true
population proportions of people by race, sex, and age in Durham, we
will utilize the 2010 Durham census data
(\url{https://www.census.gov/quickfacts/fact/table/durhamcountynorthcarolina/RHI625219\#RHI625219}).

Additionally, we will examine whether race, sex, or age are related to
the outcome of the traffic stop (whether a citation will be issued). We
hypothesize that race and the likelihood of receiving a citation are
related, with black people more likely to receive a citation upon being
stopped. We additionally hypothesize that younger people have a higher
chance of receiving a citation upon being stopped and that sex has no
significant relationship with being stopped in traffic.

It has 29 variables and 323147 observations, and each observation in the
data set is an individual police stop recorded in Durham during 2001 to
2015. A categorical variable in the data set is \texttt{subject\_race},
which describes the race of the subject involved in the traffic stop. A
discrete numerical variable in the data set is \texttt{subject\_age},
which describes the age of the subject at the time of the traffic stop.
A continuous numerical variable in the data set is \texttt{time}, which
describes the hour, minute, and second that the stop was recorded. Other
variables in the data set include \texttt{outcome}, which is what
resulted from the stop (a warning or a citation, for example);
\texttt{reason\_for\_stop}, which describes what the violation leading
to the stop was; and \texttt{search\_conducted}, whether a search of the
subject was conducted during the stop.

\hypertarget{methodology}{%
\subsection{Methodology}\label{methodology}}

\includegraphics{written-report_files/figure-latex/stop-rate-viz-1.pdf}

The variables we use to address the research question are
\texttt{subject\_race}, \texttt{subject\_sex}, and
\texttt{citation\_issued}. For these variables we filter out any unknown
and NA values. We also mutate a new categorical variable
\texttt{age\_category} based on values of \texttt{subject\_age} with the
age categories ``10-25'', ``25-40'', ``40-64'', and ``65 or older.''

To begin with, we visualize a segmented bar graph with the probability
of citation based on race below.

\includegraphics{written-report_files/figure-latex/bargraph-1.pdf}

According to the chart, it appears that Hispanics are the race with the
highest proportion of citations issued.

We also visualize a segmented bar graph with the probability of citation
based on age below.

\includegraphics{written-report_files/figure-latex/bargraph3-1.pdf}

Based on this graph, it appears that there is a roughly equal proportion
of citations issued to those in the 10-25 age group and 25-40 age group.
The proportion of citations issued then begins to decrease in the next
two age groups, with the citation proportion of the 40-64 age group
being less than the previous groups' proportions and the 65 or older
group's proportion being less than the 40-64 group's citation
proportion.

Lastly, we visualize a segmented bar graph with the probability of
citation based on sex below.

\includegraphics{written-report_files/figure-latex/bargraph2-1.pdf}

Based on this chart, the proportion of citations issued for females and
males appears to be roughly equal.

To answer our research question, we utilize the chi-square test. We
selected this test because we want to determine whether there is an
association between two variables where we have more than two samples.

We perform three chi-square tests. For the first, we ask whether there
is an association between someone's race status and whether a citation
was issued. For the second, we ask whether there is an association
between someone's age category and whether a citation was issued. For
the third, we ask whether there is an association between someone's sex
and whether a citation was issued.

With each chi-square test, we compare observed versus the expected
counts that we would expect if each \(H_0\) were true. If these total
differences are ``large enough,'' then we reject the null hypothesis. We
will perform each chi-square test at the \(\alpha = 0.05\) significance
level.

\hypertarget{results}{%
\subsection{Results}\label{results}}

We first investigated the research question in reference to stop rates.
Our exploratory data analysis indicated that black people appeared to be
stopped at a disproportionately higher rate compared to their proportion
within the Durham County population. We decided to check if this
difference was statistically significant through the below test:

Let \(\rho\) equal the true proportion of stopped drivers who were black
within Durham County.

\(H_0: \rho = 0.369\). The true proportion of stopped drivers who were
black within Durham County is equal to the true proportion of black
people within Durham County (0.369).

\(H_A: \rho > 0,369\). The true proportion of stopped drivers who were
black within Durham County is greater than the true proportion of black
people within Durham County.

\(\alpha\) = 0.05

Conditions:

\begin{enumerate}
\def\labelenumi{\arabic{enumi}.}
\item
  Independence of Outcomes - Although the 10\% condition is not met
  (since the data is an attempted census), it is reasonable to assume
  that one traffic stop does not affect the likelihood of another
  traffic stop, meaning the outcomes are independent.
\item
  Sample Size: 323147 \textgreater{} 30.
\end{enumerate}

Conditions met. Proceed with a one-proportion t-test.

\begin{verbatim}
## # A tibble: 1 x 3
##   estimate statistic p.value
##      <dbl>     <dbl>   <dbl>
## 1    0.559    50218.       0
\end{verbatim}

\ldots{}

We then investigated the second element of our research question and
conducted a series of chi-square tests of independence to determine if a
person's race or sex is associated with a higher chance of receiving a
citation upon being stopped.

\(H_0:\) Race and the likelihood of receiving a citation upon being
stopped are not associated.

\(H_A:\) Race and the likelihood of receiving a citation upon being
stopped are associated.

\(\alpha\) = 0.05.

\begin{verbatim}
## # A tibble: 1 x 3
##   statistic chisq_df p_value
##       <dbl>    <int>   <dbl>
## 1     2785.        4       0
\end{verbatim}

The chi-squared test for independence outputted a statistic of 2785.354.
The distribution of the test statistic is a chi-squared distribution,
which is unimodal and right-skewed with 4 degrees of freedom.

Since our p-value of 0 is less than our alpha of 0.05, we reject the
null hypothesis. There is sufficient evidence to indicate that race and
the likelihood of receiving a citation upon being stopped are
associated.

We then tested if a driver's sex is associated with the likelihood of
receiving a citation.

\(H_0:\) Race and the likelihood of receiving a citation upon being
stopped are not associated.

\(H_A:\) Race and the likelihood of receiving a citation upon being
stopped are associated.

\(\alpha\) = 0.05.

\begin{verbatim}
## # A tibble: 1 x 3
##   statistic chisq_df p_value
##       <dbl>    <int>   <dbl>
## 1  0.000593        1   0.981
\end{verbatim}

The chi-squared test for independence outputted a statistic of 0.001.
The distribution of the test statistic is a chi-squared distribution,
which is unimodal and right-skewed with 1 degree of freedom.

Since our p-value of 0.981 is greater than our alpha of 0.05, we fail to
reject the null hypothesis. The data does not provide sufficient
evidence to indicate that sex and the likelihood of receiving a citation
upon being stopped are associated.

By itself, the chi-squared test only provides evidence for the
association of two variables but does not inform us of the exact nature
of the association. In order to investigate the exact nature of the
association between race and the likelihood of receiving a citation, we
created a logistic regression model. We also added age as a predictor on
the model to control for the effect of age on citations. However, we
excluded sex from the model, as the above chi-squared test for
independence indicated that the data does not provide a statistically
significant association between a subject's sex and the likelihood of
receiving a citation.

Conditions of Logistic Regression:

\begin{enumerate}
\def\labelenumi{\arabic{enumi}.}
\item
  Independence - Each traffic stop is independent of other traffic
  stops; one traffic stop resulting in a citation does not affect the
  likelihood that other traffic stops result in citations.
\item
  Linearity - Below, we have depicted scatterplots of the relationship
  between age and the log-odds of receiving a citation, faceting by race
  to isolate the linear predictor. The Linearity Assumption is met
  because there is a linear relationship between the age of a subject
  and the log-odds of receiving a citation when other predictors (race
  in this context) are held constant.
\end{enumerate}

\includegraphics{written-report_files/figure-latex/linearity2-1.pdf}

Conditions met. Proceed with a logistic regression model.

\begin{verbatim}
## # A tibble: 6 x 5
##   term                               estimate std.error statistic   p.value
##   <chr>                                 <dbl>     <dbl>     <dbl>     <dbl>
## 1 (Intercept)                         0.393    0.0122      32.2   4.74e-227
## 2 subject_raceasian/pacific islander  0.00369  0.0293       0.126 9.00e-  1
## 3 subject_raceblack                  -0.202    0.00801    -25.2   4.27e-140
## 4 subject_racehispanic                0.359    0.0124      28.9   4.21e-183
## 5 subject_raceother                  -0.143    0.0567      -2.52  1.18e-  2
## 6 subject_age                        -0.00739  0.000279   -26.5   1.32e-154
\end{verbatim}

The logistic regression model has outputted the following equations to
predict the likelihood of receiving a citation:

Predicted Citation Log-Odds = 0.393 + 0.004 * subject\_raceasian/pacific
islander - 0.007 * age

Predicted Citation Log-Odds = 0.393 - 0.202 * subject\_raceblack - 0.007
* age

Predicted Citation Log-Odds = 0.393 + 0.359 * subject\_racehispanic -
0.007 * age

Predicted Citation Log-Odds = 0.393 - 0.143 * subject\_raceother - 0.007
* age

This model yields a few key conclusions:

\begin{enumerate}
\def\labelenumi{\arabic{enumi}.}
\item
  Holding age constant, we expect the odds that a Hispanic person will
  receive a citation upon being stopped by police in Durham County to be
  1.4318968 times the odds that a white person will receive a citation
  upon being stopped by police. The coefficient is also statistically
  significant (p-value \textless{} 0.01), meaning there is less than a
  1\% chance such a coefficient or more extreme would be found in the
  data if race and the likelihood of receiving a citation were not
  associated.
\item
  Holding age constant, we expect the odds that a black person will
  receive a citation upon being stopped by police in Durham County to be
  0.8170949 times the odds that a white person will receive a citation
  upon being stopped by police. The coefficient is also statistically
  significant (p-value \textless{} 0.01).
\item
  Holding race constant, for every additional year of age, we expect the
  odds of receiving a citation upon being stopped to multiply by
  0.9926402. The coefficient is also statistically significant (p-value
  \textless{} 0.01), indicating that the data does provide sufficient
  evidence that the driver's age and the likelihood of receiving a
  citation are associated (slope ≠ 0).
\end{enumerate}

The implications of this model will be further discussed in the
``Discussion'' section of the report.

Fail to reject null\ldots{}

\hypertarget{discussion}{%
\subsection{Discussion}\label{discussion}}

Though black people are disproportionately more likely to be stopped for
a traffic violation, they were not the most likely to receive a citation
upon being stopped--Hispanics were the most likely to receive a citation
after being stopped.

\end{document}
